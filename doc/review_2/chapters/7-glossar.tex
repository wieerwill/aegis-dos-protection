\documentclass[../review_2.tex]{subfiles}
\graphicspath{{\subfix{../img/}}}
\begin{document}

        %\chapter{Glossar}\thispagestyle{fancy}

        % polling
        % NIC
        % Thread
        % RX/TX Queues
        % Vtable
        % Middlebox Mitigation Box
        
        %\begin{description}
 
        %Bitte alphabetisch ordnen
        %    \item \textbf{DDoS-Attacke:}
        %    \item \textbf{DoS-Attacke:}
        %   \item \textbf{DRoS-Attacke:}
        %   \item \textbf{Ping of Death:}
        %   \item \textbf{Ping Flood:}
        %    \item \textbf{Land-Attacke:} Bei einer LAND-Attack (LAND = Local Area Networt Denial) sendet der Angreifer ein gespooftes TCP-SYN-Packet, bei welchem er zuvor sowohl die Quell- alsu auch die Ziel-IP mit der IP des Zielsystems versehen hat. Das Opfer antwortet sich stets selbst (Loop).
        %    \item \textbf{Sockstress:} Diese Angriffsart beinhaltet eine Vielzahl an Variationen. Beim Zero Windows Connection Stress öffnet der Angreifer die Verbindung und verkündet eine Fenstergröße von 0. Das Zielsystem muss solange testen, ob etwas geht, bis der Client (= Angreifer) das Fenster vergrößert. Das Problem hierbei ist, dass die Sockets unendlich lange offen bleiben, wenn dies durch sonstige Einstellungen nicht untersagt wird. Die Grundidee des Small Window Stress ist ähnlich: Der Angreifer öffnet die Verbindung und setzt die Fenstergröße möglichst klein, zum Beispiel auf 4 Bytes. Er forderr ein Paket mit großer TCP-Payload an, wobei die Fenstergröße klein bleibt. Dies führt zu einer zunehmenden Füllung des kernel memorys, da die gesamte Anwort auf winzige (4 Byte-) Blöcke aufgeteilt werden muss.
        %    \item \textbf{SYN-FIN-Attack:}
        %   \item \textbf{SYN-Flooding:} Dieser Angriff verwendet den Verbindungsaufbau des TCP-Transportprotokolls. Der Angreifer sendet ein hohes Volumen an SYN-Pakete an das Zielsystem. Oftmals sind die IP-Adressen der Pakete gefälscht. Das Zielsystem antwortet auf jede dieser Verbindungsanforderungen. Wenn die IP-Adressen gefälscht sind, dann geht diese Antowrt in Richtung der gespooften IP-Adressse. Zusätzlich allokiert das Zielsystem Speicher. Indem der Angreifer die letzte ACK-Nachricht unterschlägt ??? %fehlt
        %    \item \textbf{SYN-Frag-Attack:} Es handelt sich bei diesem Angriff um eine Variation des SYN-Floodings %fehlt
        %    \item \textbf{Teardrop:} Bei dieser Attacke sendet der Angreifer IP-Fragmente mit überlappenden Offset-Feldern, was beim Zusammensetzen der Fragmente zu Abstürzen führen kann.
        %    \item
        %    \item \textbf{UDP-Flooding:}
        %    \item
        %\end{description}
        
\end{document}
